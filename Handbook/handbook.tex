\documentclass[twoside,openright]{report}
\usepackage[utf8]{inputenc}
\usepackage[english]{babel}
\usepackage[paperheight=9in,paperwidth=6in,vmargin=1in,lmargin=0.75in,rmargin=0.5in,bmargin=1in]{geometry}
\usepackage{titlesec}
\usepackage{ccicons}
\usepackage{fancyhdr}
\usepackage{fmtcount}
\usepackage{qrcode}
\usepackage{emptypage}
\usepackage{graphicx}
\usepackage[titletoc]{appendix}
\usepackage{tikz}
\usepackage{calc}
\usetikzlibrary{shapes.misc}
\usetikzlibrary{mindmap}
\usepackage{imakeidx}
\usepackage{nomencl}
\usepackage{cmap}
\usepackage{wasysym}
\usepackage[counter-within=chapter,headings=runin]{exsheets}

\pdfobjcompresslevel=3
\pdfcompresslevel=9

\makeindex
\makenomenclature

%% For printing acronyms with the bold first letters
%% MUA --- \Acronym{Mail User Agent}
\makeatletter
\def\wordbyword#1#2{%
\let\@mymacro#1\relax
\@wordbyword#2 \@nil
}
\def\@wordbyword#1 #2\@nil{%
\@mymacro{#1}%
\ifx\relax#2\relax
\let\next\@gobble
\else
\let\next\@wordbyword
\fi
\next#2\@nil
}
\makeatother
\def\fontify#1{\MyFont #1 }
\def\FirstLetterFont#1#2{\def\MyFont{#1}\wordbyword{\fontify}{#2}}
\newcommand{\Acronym}[1]{\FirstLetterFont{\textbf}{#1}}
%% End of the acronyms macros

\renewcommand{\nomlabel}[1]{\textbf{#1}\hfil}
\newcommand{\addnom}[2]{\nomenclature{#1}{#2}}

%% Copyright page: http://tex.stackexchange.com/a/12574/38759
\def\secondpage{\clearpage\null\vfill
\pagestyle{empty}
\begin{minipage}[b]{0.9\textwidth}
\footnotesize\raggedright
\setlength{\parskip}{0.5\baselineskip}
\noindent\begin{tabular}{l}
First printing, first edition.\\
Copyright \copyright\ \the\year\ Vitaly Repin\\[1.5cm]
ISBN 978-1514168288\\
Published by Vitaly Repin\\[1.5cm]
Email: \texttt{vitaly\_repin@fsfe.org}\\
Web: \texttt{http://digitalfree.info}\\
\end{tabular}

\vspace*{9cm}
Permission is granted to make and distribute verbatim copies of this book provided
the copyright notice and this permission notice are preserved on all copies.

Permission is granted to copy and distribute translations of this book into another
language, from the original English, with respect to the conditions on distribution
of modified versions above, provided that it has been approved by Vitaly Repin.

The book has been composed in \LaTeXe.
\end{minipage}
\vspace*{2\baselineskip}
\cleardoublepage
\rfoot{\thepage}}

\makeatletter
\g@addto@macro{\maketitle}{\secondpage}
\makeatother

%% No page numbering in the chapter pages: http://tex.stackexchange.com/a/103610/38759
\makeatletter
\renewcommand\chapter{\if@openright\cleardoublepage\else\clearpage\fi
                    \thispagestyle{empty}%
                    \global\@topnum\z@
                    \@afterindentfalse
                    \secdef\@chapter\@schapter}
\makeatother

%% Set the content of the header
\newcommand{\mymark}{}
\newcommand{\sethead}[1]{\markboth{\MakeUppercase{#1}}{}\renewcommand{\mymark}{\MakeUppercase{\chaptername}~\NUMBERstring{chapter}}}

%% QR with URL
\newcommand{\descurl}[1]{%
\qrset{link}
\qrcode[level=H]{#1}
}

%% For quizzes
\newcommand{\chk}{\item[$\Box$]}

\begin{document}
\title{Road to the Free Digital Society.\\ Course Handbook}
\date{June, 2015}
\author{Vitaly Repin}

\maketitle

\fancyfoot{}
\fancyhead[LE]{\thepage}
\fancyhead[RO]{\thepage}
\fancyhead[LO,RE]{}
\fancyhead[CE]{\mymark}
\fancyhead[CO]{\leftmark}

\pagestyle{empty}
\thispagestyle{empty}

\small
\tableofcontents
\thispagestyle{empty}

\pagestyle{fancy}

%% Separate page for the chapter title: http://tex.stackexchange.com/a/139774/38759
%% At the end of the page - box \wkurl
\newsavebox{\wkurl}
\titleformat{\chapter}[display]
{\normalfont\Large\filcenter\sffamily}
{\vspace*{\fill}
 \titlerule[1pt]%
 \vspace{1pt}%
 \titlerule
 \vspace{1pc}%
 \LARGE\MakeUppercase{\chaptertitlename}~\thechapter}
{1pc}
{\huge\scshape}
[\vspace*{\fill}\centerline{\usebox{\wkurl}}\newpage]

\savebox{\wkurl}{\descurl{http://digitalfree.info/w1}}
\chapter[Week 1. Introduction to the course. Human rights in Computing. Four Essential Freedoms. Patents]{Introduction to the course.\\ Human rights in Computing.\\ Four Essential Freedoms.\\ Patents}
\sethead{Week 1. Introduction to the course}

\section{Introduction. Human rights in computing}
\begin{itemize}
\item Most important question to judge the program: What does it do for my freedom?
\item Difference between free and proprietary software.
\item Human rights in computing. Overview.
\end{itemize}


\section[Free Software and Four Freedoms. Free GNU/Linux distros. Malicious software]{Free Software and Four Freedoms.\\ Free GNU/Linux distros.\\ Malicious software}
\nomenclature{GNU}{GNU's Not Unix}
\begin{itemize}
 \item    Software freedom: What is it?
 \item    Four essentials freedoms:
\begin{enumerate}
\setcounter{enumi}{-1}\index{Four Freedoms}
 \item        The freedom to run the program as you wish, for any purpose.
 \item        The freedom to study how the program works, and change it so it does your computing as you wish.
 \item        The freedom to redistribute copies so you can help your neighbor.
 \item        The freedom to distribute copies of your modified versions to others.
\end{enumerate}
 \item    Why software freedoms are essential.
 \item    GNU/Linux operating system. History and current status.\index{Software!GNU/Linux distributions}
 \item    Freedom as a goal. Free GNU/Linux distributions.\index{Software!Free}
 \item    Features of malicious software: Digital Restrictions Management (DRM) and Surveillance. \index{Threats!DRM} \index{Software!Malicious} \index{Software!Proprietary}
 \item    How proprietary software is promoted by its victims: Skype example.\index{Software!Skype}
 \item    Free software is a matter of liberty, not price.
\end{itemize}


\section{Computational idea patents and why they are bad. Real-life examples}
\begin{itemize}
 \item    What is computational idea patent? \index{Patents}
 \item    Why computational idea patents are bad?
\begin{itemize}
 \item        Patents and copyrights are totally different things.
 \item        Analogy with music: imagine that there are musical idea patents (e.g., for melody).
 \item        Writing a program (as well as writing a symphony): combination of new and familiar ideas.
 \item        Computational idea patents in real life: LZW and natural order recalculation examples.
 \addnom{LZW}{Lempel-Ziv-Welch}
\end{itemize}
\end{itemize}

\section{Quiz 1}
\begin{question}[type=exam]
What is the most important question to judge the program?
\begin{itemize}
\chk How much does it cost to me (what is its total cost of ownership)?
\chk What does it do for my freedom?
\chk How buggy is it?
\end{itemize}
\end{question}
\begin{solution}
What does it do for my freedom?
\end{solution}

\begin{question}[type=exam]
If the program is not free we (FSF) call it ...
\begin{itemize}
\chk  Shareware.
\chk Proprietary.
\chk SaaSS.
\end{itemize}
\end{question}
\begin{solution}
Proprietary.
\end{solution}

\begin{question}[type=exam]
What is NOT one of the four freedoms?
\begin{itemize}
\chk Run the program as you wish.
\chk Freedom to redistribute the program copies.
\chk Freedom to describe flaws in the program.
\end{itemize}
\end{question}
\begin{solution}
Freedom to describe flaws in the program.
\end{solution}

\begin{question}[type=exam]
What is the meaning of the acronym GNU?
\begin{itemize}
\chk It it not an acronym. It's another name for the wildebeest.
\chk GNU's Not Unix.
\chk Good News Unlimited.
\end{itemize}
\end{question}
\begin{solution}
GNU's Not Unix.
\end{solution}

\begin{question}[type=exam]
Why Free Software Foundation (FSF) promotes the name "GNU/Linux" contrary to just "Linux"?
\begin{itemize}
\chk The system some call "Linux" is the GNU system, combined with the kernel Linux.
\chk FSF wants GNU to be associated with Linux in order to get more power in protecting Software Freedoms.
\chk The question is wrong. FSF does not promote this name.
\end{itemize}
\end{question}
\begin{solution}
The system some call "Linux" is the GNU system, combined with the kernel Linux.
\end{solution}

\begin{question}[type=exam]
Are all the GNU/Linux software distributions free?
\begin{itemize}
\chk Yes. They are all free as in freedom. But some of them can be on sale because freedom is not about price at all.
\chk No. There is only one free GNU/Linux software distribution in the world and it is distributed by FSF. All other GNU/Linux software distros contain proprietary components and can not be defined as free.
\chk No. There are few free GNU/Linux software distributions. FSF promotes them.
\end{itemize}
\end{question}
\begin{solution}
No. There are few free GNU/Linux software distributions. FSF promotes them.
\end{solution}


\begin{question}[type=exam]
You wrote a program and in the process implemented an idea. After publishing your application you found that this idea had been patented previously (before you started to even think about it). What are the dangers?
\begin{itemize}
\chk No danger because you have invented this idea independently from the patent owner.
\chk You can be sued by the patent owner. Your users can be sued also.
\chk You have committed a criminal offense described in the US penal code. There is a risk of being arrested by US police when you cross the USA border.
\end{itemize}
\end{question}
\begin{solution}
You can be sued by the patent owner. Your users can be sued also.
\end{solution}

\begin{question}[type=exam]
What a supporter of free software is expected to think about an idea to use Skype as a collaboration tool and to put the Skype user id in his or her email signature?
\begin{itemize}
\chk Sounds cool. I want to communicate with the people via Skype, a lot of my friends use it. I want them to know my Skype credentials as they will be able to use Skype to reach me.
\chk This idea is very bad. Skype is proprietary software and using it is a threat for the freedom. By putting my Skype username in the signature I will do even worse --- I will promote the usage of the non-free software among the people I cooperate with.
\chk This idea sounds bad. Skype is owned by Microsoft and this corporation should be punished for their unethical behaviour and creation of the non-free software.
\end{itemize}
\end{question}
\begin{solution}
This idea is very bad. Skype is proprietary software and using it is a threat for the freedom. By putting my Skype username in the signature I will do even worse --- I will promote the usage of the non-free software among the people I cooperate with.
\end{solution}

\begin{question}[type=exam]
What are the malicious features of Adobe Flash player?
\begin{itemize}
\chk Digital Restrictions Management (DRM) and user surveillance feature where one site was able to write data into the Flash Player and another site could interrogate the Flash Player thus enabling sites to cross-identify a user.
\chk Digital Restrictions Management (DRM) and adware.
\chk Adobe Flash Player constantly reports the user's IP address to the Adobe headquarters.
\end{itemize}
\end{question}
\begin{solution}
Digital Restrictions Management (DRM) and user surveillance feature where one site was able to write data into the Flash Player and another site could interrogate the Flash Player thus enabling sites to cross-identify a user.
\end{solution}

\begin{question}[type=exam]
Imagine that in order to promote the progress of symphonic music the European governments of XVIII century established a patent system on music themes. How would it influence the work of composers?
\begin{itemize}
\chk Composers would become richer and would be able to be more creative because of the need to invent the new themes in the music all the time.
\chk No influence at all.
\chk It would be harder for composers to create rich symphonies because every new piece of art reuses the ideas developed by previous artists in one way or the other
\end{itemize}
\end{question}
\begin{solution}
It would be harder for composers to create rich symphonies because every new piece of art reuses the ideas developed by previous artists in one way or the other
\end{solution}



\vfill

\section{Human rights in computing. Mindmap}

\newcommand{\myhref}[2]{#2}
\definecolor{light-gray}{gray}{0.97}

\pgfdeclarelayer{background}
\pgfsetlayers{background,main}
\begin{center}
\rotatebox{90}{\resizebox{!}{.7\textheight}{\begin{tikzpicture}[mindmap,
every node/.style={concept,execute at begin node=\hskip0pt},
root concept/.append style={
concept color=black, fill=white, line width=1ex, text=black,minimum size=3cm,text width=3cm,font=\large\scshape},
text=black,
concept color=light-gray, grow cyclic,
level 1/.append style={level distance=3.7cm,sibling angle=90},
level 2/.append style={level distance=3cm,sibling angle=45},
level 3/.append style={level distance=2.3cm,sibling angle=45},
remark/.style={
rectangle,
sloped,
minimum size=4mm,
very thin,
color=white,
text=black,
font=\itshape
}]
\node[root concept] (root) {Human rights in computing}
  child [concept] {node {\myhref{https://www.gnu.org/philosophy/free-software-intro.html}{Free software movement}}
  	child [concept] {node {Freedom for people using software}}
	child [concept] {node (fsf) {\myhref{https://groups.google.com/forum/\#!msg/net.unix-wizards/8twfRPM79u0/1xlglzrWrU0J}{Start: September, 1983}}}
  }
 child [concept] {node (gnulnx) {\myhref{https://www.gnu.org/gnu/linux-and-gnu.html}{GNU/Linux}}
  	child [concept] {node {\myhref{http://en.wikipedia.org/wiki/Unix}{UNIX}-like system}}
	child [concept] {node {\myhref{http://www.gnu.org/}{GNU's Not Unix}}}
	child [concept] {node {1992: First free OS}
		child [concept] {node {Almost all OS is ready}}
		child [concept] {node {\myhref{http://web.archive.org/web/20110721105526/http://www.kernel.org/pub/linux/kernel/Historic/old-versions/RELNOTES-0.12}{Linux kernel became free}}}
	}
	child [concept] {node {Nonfree sw on top of GNU/Linux limits your freedom}
		child [concept] {node {Most \myhref{http://www.gnu.org/distros/}{distros} contain nonfree sw}}
	}
  }
  child [concept] { node (fourfr) {Four Freedoms}
	child [concept] { node {3: Distribute modified versions}}
	child [concept] { node {2: Redistribution}
       		child [concept] {node {Basic moral rule - help your friends}}
        }
	child [concept] { node {1: Study and change the sources}}
        child [concept] { node {0: Run the program as you wish}}
  }
  child [concept] { node {How to judge the program?}
	child [concept] {node (freesw) {\myhref{https://www.gnu.org/philosophy/free-sw.html}{Free software}}
		child [concept] {node {User controls the program}}
	}
        child [concept] { node {What it does for my freedom?}
		child [concept] { node {\myhref{https://www.gnu.org/philosophy/selling.html}{"Free" is not about price at all}}}
	}
	child [concept] {node {Nonfree software}
		child [concept] {node {Program controls the user}}
		child [concept, sibling angle=45] {node {Examples: Gratis but nonfree}
			child [concept, sibling angle=45] {node {Adobe Flash Player}
				child [concept] {node {DRM: restricts the user}}
				child [concept] {node {Surveillance: Super-Cookies}}
				child [concept] {node {Malicious}}
			}
			child [concept, sibling angle=45] {node {\myhref{http://stallman.org/skype.html}{Skype}}
				child [concept] {node {Promotes nonfree software}}
			}
			child [concept, sibling angle=45] {node {\myhref{http://www.gnu.org/philosophy/proprietary-surveillance.html}{Angry Birds}}}
		}
	}
  };
 \begin{pgfonlayer}{background}
  	\draw [concept connection] (fourfr) edge node[remark,above=3.5pt] {Gives} node [remark,below=3.5pt] {Four Freedoms} (freesw);
	\draw [concept connection] (fsf) edge node[remark,above=3.5pt] {Goal: develop} node [remark,below=3.5pt] {entirely free OS} (gnulnx);
  \end{pgfonlayer}
\end{tikzpicture}}}
\end{center}



\savebox{\wkurl}{\descurl{http://digitalfree.info/w2}}
\chapter[Week 2. Threats to the Free Digital Society]{Threats to\\ the Free Digital Society}
\sethead{Week 2. Threats to the Free Digital Society}
\section[Digital Society: Inclusion or Escape? Surveillance, censorship, restricting the users]{Digital Society: Inclusion or Escape?\\ Surveillance, censorship, restricting the users}

\begin{itemize}
  \item     Shall we aim for inclusion to or escape from the Digital Society?\index{Digital Society}
  \item     What are the threats to the Digital Society? Overview.
  \item     Threat 1: Surveillance.
\begin{itemize}
  \item         Surveillance and Democracy: how are they compatible? \index{Threats!Surveillance}
  \item         Who watches the watchmen? Why society needs whistleblowers.
  \item         Common engineering practice of the day: keep all the data you can.
  \item         Three ways to spy on people. Examples and the ways to react:
\begin{itemize}
  \item             Through their own systems.
  \item             Through the systems they use but don't own (e.g., phone networks).
  \item             Surveillance systems which are designed specifically for this activity.
\end{itemize}
  \item         Data from business surveillance systems is available to the state.
\end{itemize}
  \item     Threat 2: Censorship. \index{Threats!Censorship}
\begin{itemize}
  \item         Censorship in Internet: "filtration".
  \item         Country examples (Europe, Asia, ...).
  \item         Propaganda methods: using disgusting things (e.g., child pornography) to advocate censorship tools.\index{Threats!Propaganda}
\end{itemize}
  \item     Threat 3: Data formats that restrict their users.
\begin{itemize}
  \item         Digital handcuffs.
  \item         DRM: Digital Restrictions Management. \index{Threats!DRM}
\addnom{DRM}{Digital Restrictions Management / Digital Rights Management}
  \item         Reasons why secret data formats exist.
\end{itemize}
\end{itemize}

\section[Threats from software that the user do not control]{Threats from software that\\ the user do not control}
\begin{itemize}
 \item    There are only two options: the users controls the program OR the program controls the users.
 \item    Backdoors and Universal Backdoors. \index{Threats!Backdoor}
 \item    Real-life examples: Amazon Swindle, portable phones. \index{Swindle} \index{Mobile Phones}
\end{itemize}


\section[Threats from Service as a Software Substitute]{Threats from Service\\ as a Software Substitute}
\begin{itemize}
 \item    SaaSS: new way to lose control over your computing.\index{Threats!SaaSS}
 \item    SaaSS: how it works. Technical explanation.
 \item    SaaSS is equivalent to running non-free program with universal backdoor which makes it proprietary malware. \index{Threats!Backdoor}
 \item    Paradoxical relationships between SaaSS and free/non-free software.
\begin{itemize}
 \item        If server software is free, it does not help the users. It benefits only the server owner.
 \item        If server software is non-free then server owner does not control it. Somebody else controls it. The users do not benefit in either of the situations.
\end{itemize}
 \item    Not all the services are SaaSS:
\begin{itemize}
 \item        Sometimes you can't fully control your computing --- e.g., when you are communicating with other people.
 \item        If your task can be done if you have powerful computer and required software, then server-based solution is SaaSS. Otherwise it is not.
\end{itemize}
 \item    Examples of SaaSS: Translation and Speech Recognition services.
\end{itemize}


\section{War on Sharing. Precarity}\index{Threats!Precarity}
\begin{itemize}
 \item    What is sharing and why it is good.
 \item    Attack on sharing is an attack on social cooperation.
 \item    Two kinds of works:
\begin{itemize}
 \item        Works we need to do practical jobs (e.g. reference works, educational works): should be free.\index{Authorship}
 \item        Others (e.g. entertainment, opinion, art) should be freely shareable.
\end{itemize}
 \item    Ways to attack sharing:
\begin{itemize}
 \item        Laws which forbid sharing.
 \item        Propaganda. Terms like "Piracy" and "Theft":\index{Sharing}\index{Piracy}\index{Threats!Propaganda}
 \item            Reject propaganda terms! Attacking ship is very bad, sharing is good.
 \item            Legally speaking, copyright infringement is never theft. Sometimes it's a crime, sometimes not. But is it NEVER a theft.
 \item        Publishing of the works using secret formats with the purpose to restrict the public:
 \item            DVD "conspiracy": if you want to manufacture DVD player, you need to join the conspiracy. No competition in this important part of DVD player functionality!
 \addnom{DVD}{Digital Versatile Disc}
 \item        DMCA: Digital Millennium Copyright Act: \index{Threats!DMCA}
\addnom{DMCA}{Digital Millennium Copyright Act}
\begin{itemize}
 \item            Banning distribution of free software which allows to read DVDs in the secret format.\index{Software!Free}
 \item            Similar laws in EU and other countries.
\end{itemize}
 \item        DRM: Digital Restrictions Management: \index{Threats!DRM}
\begin{itemize}
 \item            Introduced by AACS (Advanced Access Content System) "conspiracy".
 \addnom{AACS}{Advanced Access Content System}
 \item            Example of AACS power: they managed to ban analog video outputs.
\end{itemize}
 \item        Disconnect people from Internet when they are \emph{accused} of sharing (without a trial):\index{Sharing}
\begin{itemize}
 \item            Crashing of the basic principle of justice --- no punishment without a trial.
\end{itemize}
\end{itemize}
 \item    Controlling what our technology does is a worse attack on society than controlling prices. It shall be treated as more grave crime.
 \item    Usage of Internet is precarious:
\begin{itemize}
 \item        You need to cooperate with ISP, DNS registrar, other companies in order to do your tasks.\index{ISP}\index{Threats!Precarity}
\addnom{ISP}{Internet Service Provider}
\addnom{DNS}{Domain Name System}
 \item        Typically the contracts you sign are so, that they can be canceled for any reason by service provider.
 \item        Example: Dirty Tricks campaign against WikiLeaks.\index{WikiLeaks}\index{Freedom Fighters}
 \item        Analogy: imagine that government convinces your phone or electrical company that it's in their interest to stop serving you.
 \item        Regulation is needed like for other public utilities: no right to disconnect if I pay according to the contract, no way to discriminate single customer etc.
\end{itemize}
\end{itemize}

\section{Quiz 2}

\begin{question}[type=exam]
Is it a good or bad thing to participate in the digital society? Select one answer.
\begin{itemize}
\chk Yes, sure. New digital technologies improve people's living conditions dramatically and we should do our best to let everybody to benefit from the new technologies.
\chk No, not at all. New digital technologies make the people less smart, they restrict abilities of the human brain to think. We need to avoid them and focus on developing our brains instead.
\chk It depends on how digital society is set up and how does it respect people's freedom.
\end{itemize}
\end{question}
\begin{solution}
It depends on how digital society is set up and how does it respect people's freedom.
\end{solution}

\begin{question}[type=exam]
Which of the following is NOT an example of surveillance threat to the people's freedom?
\begin{itemize}
\chk Systems to record the user's location using the data from GSM or CDMA base stations where user's phone was registered.
\index{GSM}
\index{CDMS}
\nomenclature{GSM}{Groupe Spécial Mobile}
\nomenclature{CDMA}{Code Division Multiple Access}
\chk Portable GPS tracker gadget which tracks your geo-location and stores it in its memory. GPS track can be downloaded from the gadget through USB or RS-232 interfaces.
\index{USB}
\index{RS-232}
\index{GPS}
\nomenclature{GPS}{Global Positioning System}
\nomenclature{USB}{Universal Serial Bus}
\chk Companies monitoring the behavior of their users via web sites (e.g., cookies) for advertising purposes.
\end{itemize}
\end{question}
\begin{solution}
Portable GPS tracker gadget which tracks your geo-location and stores it in its memory. GPS track can be downloaded from the gadget through USB or RS-232 interfaces.
\end{solution}

\begin{question}[type=exam]
Several examples of censorship in different countries were presented in the video. Which country was NOT mentioned in this context?
\begin{itemize}
\chk Turkey
\chk Finland
\chk Saudi Arabia
\end{itemize}
\end{question}
\begin{solution}
Saudi Arabia
\end{solution}

\begin{question}[type=exam]
What are "digital handcuffs"?
\begin{itemize}
\chk Modern handcuffs for the prisoners with escape records. They have surveillance functionality implemented using CDMA technology.
\chk The situation when software users are restricted by the secret data format which prevents data interchange with some of the other programs.
\chk The situation when the user is forced to use the proprietary software because there is no free software with the similar functionality available.
\end{itemize}
\end{question}
\begin{solution}
The situation when software users are restricted by the secret data format which prevents data interchange with some of the other programs.
\end{solution}

\begin{question}[type=exam]
"Software as a Service Substitute" means losing the control on your computing as it is done on somebody's else server. What is the difference between Wikipedia and Speech Recognition services from end-user point of view?
\begin{itemize}
\chk No difference. Both process user's data on their side and both should be avoided because the users is loosing the control on his/her computing.
\chk By contributing to Wikipedia the user helps it to do its computing and this is expected by the user. By using Speech Recognition service user's own computing is done by the server.
\chk Wikipedia is a service to share the data with other users in a free way. Speech Recognition service is proprietary solution therefore it is bad idea to use it.
\end{itemize}
\end{question}
\begin{solution}
By contributing to Wikipedia the user helps it to do its computing and this is expected by the user. By using Speech Recognition service user's own computing is done by the server.
\end{solution}

\begin{question}[type=exam]
What is mr. Stallman's opinion on the piracy? Select the best answer.
\begin{itemize}
\chk Attacking ships is bad. But it has nothing in common with sharing which is good. It's morally bad to try to associate something which is good (sharing) with something which is bad (attacking ships).
\chk Piracy is a good thing and we should all be pirates. It is in the nature of the human being to share things.
\chk Piracy is sometimes good and sometimes bad. It depends on the nature of the works which are shared (e.g., sharing of the fiction books can be sometimes considered bad while sharing of the engineering books is always good).
\end{itemize}
\end{question}
\begin{solution}
Attacking ships is bad. But it has nothing in common with sharing which is good. It's morally bad to try to associate something which is good (sharing) with something which is bad (attacking ships).
\end{solution}

\begin{question}[type=exam]
What Precarity threat is about?
\begin{itemize}
\chk Everybody needs to be educated about the dangers of the digital society but in reality only the minor percentage of the population has this knowledge.
\chk Any activity in the Internet requires cooperation from a lot of stakeholders --- ISP, hosting providers etc. Governments can use these stakeholders to disconnect the service (e.g., web site) they don't like from the Internet by putting pressure on those stakeholders. It is possible because a lot of service providers have a freedom to break the contract with the user in any time.
\chk Children are becoming the victims of the violent content accessible in the Internet. As a result, school shooting occur. New policy towards the way children can use Internet should be developed.
\end{itemize}
\end{question}
\begin{solution}
Any activity in the Internet requires cooperation from a lot of stakeholders --- ISP, hosting providers etc. Governments can use these stakeholders to disconnect the service (e.g., web site) they don't like from the Internet by putting pressure on those stakeholders. It is possible because a lot of service providers have a freedom to break the contract with the user in any time.
\end{solution}

\begin{question}[type=exam]
What is "universal backdoor"? Select one answer.
\begin{itemize}
\chk The interface through which somebody else than the user can install any software changes in the user's system without user's permission.
\chk Method of bypassing normal authentication, securing unauthorized remote access to a computer, obtaining access to plaintext, and so on, while attempting to remain undetected.
\chk The name of the method to unencrypt DRM-protected movies in some of DVD players.
\end{itemize}
\end{question}
\begin{solution}
The interface through which somebody else than the user can install any software changes in the user's system without user's permission.
\end{solution}

\begin{question}[type=exam]
Which parts of "dirty tricks" campaign against WikiLeaks was discussed in the lectures?
\begin{itemize}
\chk CIA used sexual entrapment against Julian Assange.
\nomenclature{CIA}{Central Intelligence Agency}
\chk US government tried to convince the companies which provided services to WikiLeaks (e.g., eCommerce services) that it's in their best interests to stop serve WikiLeaks.
\chk Governments run defamation campaign against WikiLeaks trying to portrait it as untrustworthy source of information.
\end{itemize}
\end{question}
\begin{solution}
US government tried to convince the companies which provided services to WikiLeaks (e.g., eCommerce services) that it's in their best interests to stop serve WikiLeaks.
\end{solution}

\begin{question}[type=exam]
Why SaaSS is an equivalent of proprietary malware? Select one answer.
\begin{itemize}
\chk If you use SaaSS your computing is done by the server. Server owner can misuse your data.
\chk SaaSS has the same effect as running non free programs with universal backdoor (because server owner can change server software in any time and it changes the way user's computing is done).
\chk SaaSS is proprietary solution. Proprietary solution is always malware.
\end{itemize}
\end{question}
\begin{solution}
SaaSS has the same effect as running non free programs with universal backdoor (because server owner can change server software in any time and it changes the way user's computing is done).
\end{solution}



\savebox{\wkurl}{\descurl{http://digitalfree.info/w3}}
\chapter[Week 3. Copyright and Copyleft: How to make software free?] {Copyright and Copyleft:\\ How to make software free?}
\sethead{Week 3. Copyright and Copyleft}
\section{Free Software Licenses. Introduction}
\begin{itemize}
\item    Why does a free program have a license?\index{Licenses}\index{Software!Free}
\begin{itemize}
 \item        Everything which is written (including programs) have a license automatically. "Default" (silent) license does not allow people to modify and distribute programs.
 \item        Typical proprietary license: a contract which restricts users even more than copyright laws.
 \item        Free software license: gives back to the users their rights which were taken by copyright laws.
\end{itemize}
 \item    Copyleft and non copyleft free software licenses.\index{Licenses!Copyleft}\index{Licenses!Copyright}
\begin{itemize}
 \item        In order for the license to be free software license it shall give the users four essential freedoms (see lecture 1.2). There are many ways to do it which explains why there are many free software licenses.
 \item        Big dichotomy among free software licenses: copyleft and non copyleft.
 \item        Copyleft does not allow the user to take free software and make it non free (even after making modifications to the software).
\begin{itemize}
 \item            The problem is real. E.g., \TeX. Its UNIX implementation was made non free while original Knuth's\index{Knuth, Donald} implementation was free (but distributed under non copyleft license).\index{Software!TeX@\TeX}
 \item            Copyleft idea: to give the freedom to the "next" users of the program.
\end{itemize}
\end{itemize}
\end{itemize}

\section{GNU GPL and Affero GPL: Copyleft licenses}
\begin{itemize}
 \item     GNU General Public License: GPL.\index{Licenses!GPL}\index{Licenses!Copyleft}
\begin{itemize}
 \item         Copyleft license
 \item         Originally GPL was developed as a license to distribute GNU system under. GPL 3 is designed for other kind of work also.\index{Licenses!GPL}
 \item         GPL prohibits a lot of different means to make free software non-free. Examples:
\begin{itemize}
 \item             \emph{Release only binaries.} GPL requires source code to be available.
 \item             \emph{Change of the license.} GPL requires to keep the license.
 \item             \emph{Patent.} It's not allowed to use patents to limit the freedom of software distributed under GPL.
\end{itemize}
\end{itemize}
 \item     Other free software licenses are non copyleft. Some of them are very weak ("pushover licenses" allow almost everything). There are also weak copyleft licenses. E.g., Mozilla Public License (MPL):\index{Licenses!Pushover}
\begin{itemize}
 \item         Files which were received under MPL shall be distributed under the same license (MPL) but new files can be added with different license.\index{Licenses!MPL}
\addnom{MPL}{Mozilla Public License}
 \item         Way to make free software non free: add new source code files under other license and call the new subroutines defined there from original files.
\end{itemize}
 \item     GNU Affero General Public License: AGPL.\index{Licenses!AGPL}
\begin{itemize}
 \item         Secondary effect of copyleft: because modifications are free, developer of the original version cat take them and incorporate to the original software. This is the way to contribute to the community's shared knowledge.
 \item         Installing modified GPL software on the server is not its distribution. Hence, it's not needed to make the source code available. Therefore second effect of copyleft does not work.
 \item         AGPL addresses this problem. It requires source code to be available if the software runs on server and is available to others.
 \item         Relation to SaaSS (see also lecture 2.3):
\addnom{AGPL}{GNU Affero General Public License}
\addnom{GPL}{GNU General Public License}
\begin{itemize}
 \item             No contradiction with "SaaSS is bad" logic: there are many server software which is not SaaSS and which needs AGPL protection.\index{Threats!SaaSS}\index{Licenses!AGPL}
 \item             No license can make SaaSS ethical. The only way to solve the problem is to take server code and execute it by the user.
\end{itemize}
\end{itemize}
\end{itemize}


\section{Non copyleft and weak copyleft licenses}
\begin{itemize}
 \item    Sometimes there are advantages in using non copyleft license.\index{Licenses!Weak copyleft}\index{Licenses!Pushover}
\begin{itemize}
 \item        \emph{Ogg/Vorbis player.} It was considered as very important for all music players to be able to support this free format. That's why Ogg/Vorbis codec was distributed under non copyleft license: Apache 2.0 License.\index{Data Formats!Ogg/Vorbis}
 \item        Recommended non copyleft license to use: Apache 2.0. Reason: patent protection \index{Licenses!Apache 2.0}
\end{itemize}
 \item    BSD licenses: original and modified (revised):
\begin{itemize}
 \item        These licenses are very different. Don't state "The code is distributed under BSD license". The question is --- \emph{which} BSD license?
 \item        Important difference: advertising requirements.
 \item        Original BSD license is not recommended to be used to distribute new software under.\index{Licenses!BSD}
 \addnom{BSD}{Berkeley Software Distribution}
\end{itemize}
 \item    Other free software licenses:
\begin{itemize}
 \item        X11, Expat licenses are not "MIT licenses". They are different licenses. Refer to each one specifically.\index{Licenses!MIT}\index{Licenses!X11}
\addnom{MIT}{Massachusetts Institute of Technology}
 \item        FSF: Various Licenses and Comments about them \addnom{FSF}{Free Software Foundation}
 \item        FSF: How to choose a license for your own work
\end{itemize}
\end{itemize}

\section{Quiz 3}
\begin{question}[type=exam]
What is the primary purpose of free software license?
\begin{itemize}
\chk To allow free distribution of modifications made to the free software. Therefore to contribute to the community's shared knowledge.
\chk To give the user back their freedom taken by copyright law.
\chk To free the developers of the software from any obligations and claims which can be risen by their users.
\end{itemize}
\end{question}
\begin{solution}
To give the user back their freedom taken by copyright law.
\end{solution}


\begin{question}[type=exam]
Which problem is addressed by Affero GPL which is not in scope of GPL?
\begin{itemize}
\chk If the modified software is executed in the server and is used by others, modifications of the server software shall also be available to those "others".
\chk Software libraries distributed under GPL force the packages which use them to be licensed under GPL as well. AGPL addresses this issue by making copyleft conditions weaker than in original GPL.
\chk GNU system needed non copyleft licenses to accompany copyleft (which is GPL) because some of the GNU projects wanted to use non copyleft licenses. AGPL is an answer to this demand.
\end{itemize}
\end{question}
\begin{solution}
If the modified software is executed in the server and is used by others, modifications of the server software shall also be available to those "others".
\end{solution}


\begin{question}[type=exam]
Is Mozilla Public License (MPL) a copyleft and why? Select one answer.
\begin{itemize}
\chk Yes. It makes it impossible to make free program non free.
\chk No. It is possible to make free program non free if original program is released under MPL.
\chk Yes. Because it requires the original source code to be distributed under MPL license as well.
\end{itemize}
\end{question}
\begin{solution}
No. It is possible to make free program non free if original program is released under MPL.
\end{solution}


\begin{question}[type=exam]
Which non copyleft license is recommended by FSF to be used with projects which needs to be distributed under non copyleft freesoftware license?
\begin{itemize}
\chk X11 License.
\chk Apache 2.0 License.
\chk GNU AGPL.
\end{itemize}
\end{question}
\begin{solution}
Apache 2.0 License.
\end{solution}


\begin{question}[type=exam]
What was the reason for distribution Ogg/Vorbis codec under non copyleft license?
\begin{itemize}
\chk It was very important to extend the range of music players supporting Ogg/Vorbis format.
\chk It was decided to test and measure the effect of selecting non copyleft license on software distribution coverage.
\chk It was a huge mistake --- developers of the codec were not aware of the disadvantages of non copyleft licenses as explained in our lectures.
\end{itemize}
\end{question}
\begin{solution}
It was very important to extend the range of music players supporting Ogg/Vorbis format.
\end{solution}


\begin{question}[type=exam]
What is called "Secondary Effect of Copyleft"?
\begin{itemize}
\chk Copyleft gives users the Second Freedom --- the freedom to study how the program works, and change it so it does their computing as they wish.
\chk Copyleft prohibits limiting users freedom and as secondary effect proprietary software developers don't like to use and modify the software which is distributed under copyleft.
\chk Modifications to the program distributed under copyleft license can be included into the original program. Therefore copyleft facilitates contribution to the communitys shared knowledge.
\end{itemize}
\end{question}
\begin{solution}
Modifications to the program distributed under copyleft license can be included into the original program. Therefore copyleft facilitates contribution to the communitys shared knowledge.
\end{solution}


\begin{question}[type=exam]
What is called "pushover license" and why?
\begin{itemize}
\chk These free software licenses are very weak non copyleft licenses. They are called this way because they permit almost everything.
\chk This is another name for copyleft licenses. It emphasizes the fact that these licenses are pushing the freedom to the users.
\chk This name is used for proprietary licenses. It emphasizes the fact that proprietary software is pushed to the users with the help of these licenses.
\end{itemize}
\end{question}
\begin{solution}
These free software licenses are very weak non copyleft licenses. They are called this way because they permit almost everything.
\end{solution}


\begin{question}[type=exam]
I want to become SaaSS user. Server code is distributed under AGPL. Am I safe?
\begin{itemize}
\chk Yes. Because server code is distributed under free software license and I have access to the server software source code.
\chk No. No license can make SaaSS ethical. My computing is still done by the server and my privacy is under threat, for example.
\chk No. AGPL is not a license which can be used to protect the users from SaaSS. Other license (which is mentioned in the additional materials for this course) shall be used to achieve this effect.
\end{itemize}
\end{question}
\begin{solution}
No. No license can make SaaSS ethical. My computing is still done by the server and my privacy is under threat, for example.
\end{solution}


\begin{question}[type=exam]
Why it's a bad idea to state in the source code "This source code is distributed under BSD license"? Select one answer.
\begin{itemize}
\chk There is more than one BSD license. It's better to be specific.
\chk It's not enough to make such a one-line statement. I need to include the text of the license in every source code file.
\chk Because in this case all the users of this software will need to mention "Berkley University" in all the advertisings of their products (based on this software).
\end{itemize}
\end{question}
\begin{solution}
There is more than one BSD license. It's better to be specific.
\end{solution}


\begin{question}[type=exam]
What happens if I publish my software without any license?
\begin{itemize}
\chk My software will give freedom to the users. They will exercise four essential freedoms described in the lecture 1.2.
\chk My software will be in so called "public domain". Hence, it will give freedom to the users.
\chk My software will be automatically copyrighted. Others will not be able to exercise all the four essential freedoms described in the lecture 1.2.
\end{itemize}
\end{question}
\begin{solution}
My software will be automatically copyrighted. Others will not be able to exercise all the four essential freedoms described in the lecture 1.2.
\end{solution}





\savebox{\wkurl}{\descurl{http://digitalfree.info/w4}}
\chapter[Week 4. Works of Authorship in the Free Digital Society. History, Philosophy, Practice]{Works of Authorship in\\ the Free Digital Society.\\ History, Philosophy, Practice}
\sethead{Week 4. Works of Authorship}\index{Authorship}
\section{History of Copyright Law}
\begin{itemize}
 \item    Our basic principles are deep, technology is superficial. But it can make the very same act more or less good.
 \item    Copying in the ancient world:\index{Antiquity}
\begin{itemize}
 \item        Slow and inefficient.
 \item        No economy of scale.
 \item        Required skills: only reading and writing.
 \item        Required equipment: only ''equipment'' needed to read and write.
 \item        Decentralized system of copying. Nothing like copyright law.
\end{itemize}
 \item    Copyright and censorship were closely related throughout the history.\index{Licenses!Copyright}\index{Threats!Censorship}
 \item    Copying in the age of printing press:
\begin{itemize}
 \item        Economy of scale.
 \item        Special expensive equipment is required.
 \item        Special skills (very different from reading and writing) are needed to operate printing press.
 \item        Centralized system of copying. Beginning of copyright.
\end{itemize}
 \item    Beginning of copyright. Copyright as a tool for industrial regulation:
\begin{itemize}
 \item        Appeared in England in XVI century.
 \item        Started from perpetual monopoly to publish certain book. Changed in XVII century: monopoly to an author (not a publisher!) for 14 years. Could be renewed once if the author was still alive.
 \item        Idea of copyright: scheme which encourages writing.
 \item        US Constitution (1788) allows congress to create a copyright system with the purpose to promote progress. Copyright can not be perpetual, only time-limited.
\end{itemize}
 \item    In the age of printing press copyright was used as an industry regulation tool. It regulated publishers and was controlled by authors. Copyright law was:\index{Printing Press}
\begin{itemize}
 \item        \emph{Mostly uncontroversial} because it restricted the publishers and not the readers. If you are not a publisher, you don't have strong reasons to object.
 \item        \emph{Easy to enforce} because it was easy to understand who published certain book. No need to visit every reader to enforce the copyright law.
 \item        \emph{Beneficial for the society} --- public traded part of its natural rights which it did not exercised to the real benefit --- more books written.
\end{itemize}
\end{itemize}

\section{Copyright in the digital age }
\begin{itemize}
 \item    Computer networks --- new advance in the technology of copying.\index{Licenses!Copyright}\index{Computer networks}
 \item    Copying became more effective and benefit of printing press in mass production almost disappeared.
 \item    Situation now reminds antiquity with the difference in efficiency of copying.
 \item    Copyright law now restricts everyone and controlled mostly by the publishers in the name of authors.
 \item    The reasons copyright was beneficial for society in the age of printing press are not valid anymore:
\begin{itemize}
 \item        \emph{No more uncontroversial} because now it restricts everybody, not only publishers.
 \item        \emph{Not easy to enforce} because the copyright law shall be enforced against everyone.
 \item        \emph{No longer beneficial.} Public wants its natural rights back because now we are able to exercise them.
\end{itemize}
 \item    Changes in copyright practices:
\begin{itemize}
 \item        Wave of copyright time extensions all over the world. Examples:
\begin{itemize}
 \item            EU: extended copyright time for sound records and textual work.
\addnom{EU}{European Union}
\addnom{USA}{United States of America}
 \item            USA: Mickey mouse copyright act (20 years extension of copyright time).\index{Threats!Mickey Mouse Copyright Act}
\end{itemize}
 \item        Real reasons to extend copyright time --- companies had valuable rights which were about to expire.
 \item        We are on a way to perpetual copyright with constant extension of copyright law. \index{Licenses!Copyright}
 \item        Mexico example: copyright validity time is 100 years after author's death.
 \item        Previously copyright was an exception and not a general rule. Not anymore. Publishers try to have total control over public through digital technologies (Digital Restrictions Management).
 \item        Attempts to take total control over public are made with non-free software --- it's not possible to do this with the help of free software.\index{Software!Free}
\end{itemize}
\end{itemize}

\section{Digital distribution: Ethical and Unethical}
\begin{itemize}
 \item     What is wrong with Amazon ''Swindle''?\index{Swindle}\index{Licenses!Copyright}\index{Authorship}
 \item     ''Swindle'' is an appropriate name for Amazon's e-reader because it swindles readers out of the traditional freedoms of readers:
\begin{itemize}
 \item         \emph{Freedom to buy a book anonymously.} Amazon requires its readers to identify themselves and maintain a list with the books which were read by the user. Existence of this list is dangerous for the fundamental human rights. But Swindle also reports to Amazon the pages which are read, sends notes and highlights to Amazon. Swindle is a complete surveillance of reading device.
 \item         \emph{Freedom to give a book to a friend, to sell it through used book store.} Swindle End User License Agreement (EULA) rejects an idea of property --- the book is owned by Amazon, not by the user! ''Swindle'' DRM and EULA together swindle readers out of this right.\index{Threats!DRM}
 \item         \emph{Freedom to keep a book as long as you wish.} Amazon can remotely delete books from Swindle and was doing it already ("1984" case).\index{Orwell, ''1984''}
 \item         Swindle has universal backdoor (autoupgrade, see lecture 2.2) which means additional restrictions can be installed later by Amazon.\index{Swindle}
\end{itemize}
 \item     Swindle is not the only malicious e-reader device: most of other e-readers are doing a lot of snooping too.
 \item     Requirements to be met in order for digital distribution to be ethical:
\begin{itemize}
 \item         No DRM (Digital Restrictions Management).\index{Threats!DRM}
 \item         No EULA --- books shall be your property.
\addnom{EULA}{End-User License Agreement }
 \item         Possibility to purchase anonymously.
\end{itemize}
 \item     Currently in most cases distribution of digital copies over Internet is ethically worse than distribution of physical copies.
 \item     The same problems arise with audio and video (e.g., DRM is used with streaming services).\index{Threats!DRM}
\end{itemize}


\section[The Why, What and How of Creative Commons]{The Why, What and How\\ of Creative Commons}
\begin{itemize} \index{Authorship}
  \item     Different Creative Commons (CC) licenses give more or less freedoms:\index{Licenses!Creative Commons}
  \addnom{CC}{Creative Commons}
\begin{itemize}
  \item         Six main CC licenses. Two of them are free.
  \item         CC BY --- push over license. CC BY SA --- copyleft license
  \item         Four other CC licenses are non-free. E.g., NC license restricts commercial usage and ND license restricts modification.
\end{itemize}
  \item     Which CC license to use for the project?
\begin{itemize}
  \item         For statements of opinion, artistic works non-free licenses are OK.
  \item         If the works shall be free (e.g., practical work, reference, educational) --- use one of two free CC licenses.
  \item         License choice is crucial --- don't delay this decision until the end of the project. Agree with project team in the very beginning!
\end{itemize}
  \item     ''This work is distributed under CC license'' --- confusing statement. \emph{Which} CC license? It makes a lot of difference.
  \item     Never distribute software under CC license! See lecture 3 for software licenses information.
\end{itemize}

\section{Quiz 4}

\begin{question}[type=exam]
Select correct statements about system of copying in antiquity:
\begin{itemize}
\chk It had no economy of scale.
\chk It was decentralized.
\chk It was subject to copyright regulation.
\end{itemize}
\end{question}
\begin{solution}
\begin{itemize}
\item It had no economy of scale.
\item It was decentralized.
\end{itemize}
\end{solution}


\begin{question}[type=exam]
Which statement about copying system in the age of printing press is NOT correct?
\begin{itemize}
\chk It had economy of scale.
\chk It was subject to copyright regulation.
\chk It was decentralized.
\end{itemize}
\end{question}
\begin{solution}
It was decentralized.
\end{solution}


\begin{question}[type=exam]
When copyright began?
\begin{itemize}
\chk In the antiquity.
\chk In age of a printing press.
\chk We were not able to trace the origins of copyright in the history.
\end{itemize}
\end{question}
\begin{solution}
In age of a printing press.
\end{solution}


\begin{question}[type=exam]
Why copyright system was mostly uncontroversial in the age of a printing press?
\begin{itemize}
\chk It was easy to enforce.
\chk It was beneficial for society.
\chk It restricted only limited amount of people --- publishers.
\end{itemize}
\end{question}
\begin{solution}
It restricted only limited amount of people --- publishers.
\end{solution}


\begin{question}[type=exam]
Is it easy to enforce copyright system in our times and why? Select the answer which is closest to what was stated in the lecture:
\begin{itemize}
\chk It is not easy because it shall be enforced against everyone which requires invading people's homes, computers and Internet connections.
\chk It is not easy because the copyright law is no more uncontroversial and a lot of people stop to respect it.
\chk It is easy because modern day technology (e.g. DRM / digital handcuffs) are used extensively nowadays.
\end{itemize}
\end{question}
\begin{solution}
It is not easy because it shall be enforced against everyone which requires invading people's homes, computers and Internet connections
\end{solution}


\begin{question}[type=exam]
Which conditions shall be met in order for distribution of digital copies to be ethical? Select all that apply:
\begin{itemize}
\chk No DRM.
\chk No EULA.
\chk Copies shall be distributed under one of free licenses (e.g., CC BY, CC BY SA or GFDL).
\nomenclature{GFDL}{GNU Free Documentation License}
\chk Distributor shall use only free software to distribute the copies.
\chk Ability to purchase anonymously
\end{itemize}
\end{question}
\begin{solution}
\begin{itemize}
\item No DRM.
\item No EULA.
\item Ability to purchase anonymously.
\end{itemize}
\end{solution}


\begin{question}[type=exam]
Which Creative Commons licenses are free? Select all the apply:
\begin{itemize}
\chk Attribution-ShareAlike 4.0 International (CC BY-SA 4.0).
\chk Attribution-NonCommercial 4.0 International (CC BY-NC 4.0).
\chk Attribution 4.0 International (CC BY 4.0).
\end{itemize}
\end{question}
\begin{solution}
\begin{itemize}
\item Attribution-ShareAlike 4.0 International (CC BY-SA 4.0).
\item Attribution 4.0 International (CC BY 4.0).
\end{itemize}
\end{solution}


\begin{question}[type=exam]
Which of the following licenses are OK for artistic works but are not good for practical works?
\begin{itemize}
\chk Attribution 4.0 International (CC BY 4.0)
\chk Attribution-NonCommercial 4.0 International (CC BY-NC 4.0)
\chk Attribution-NoDerivatives 4.0 International (CC BY-ND 4.0)
\end{itemize}
\end{question}
\begin{solution}
\begin{itemize}
\item Attribution-NonCommercial 4.0 International (CC BY-NC 4.0)
\item Attribution-NoDerivatives 4.0 International (CC BY-ND 4.0)
\end{itemize}
\end{solution}


\begin{question}[type=exam]
Somebody asks you to participate in a project and tells that results of this project (e.g., photos you made) will be distributed under Creative Commons license. You want the project results to be free. What is better to clarify first? Select all that apply:
\begin{itemize}
\chk Which CC license will be used?
\chk Will you use only free software in order to execute the project and publish it?
\chk Will the results of the project be not only free (as in "freedom") but also gratuitous for the public?
\end{itemize}
\end{question}
\begin{solution}
Which CC license will be used?
\end{solution}


\begin{question}[type=exam]
Why copyright was beneficial for society in times when it was used as a tool for industrial regulation? Select the best answer:
\begin{itemize}
\chk It was easy to enforce copyright
\chk Public traded away certain parts of its natural rights that ordinary people were not exercising and in exchange got real benefits of more books being written`
\chk It was mostly uncontroversial
\end{itemize}
\end{question}
\begin{solution}
Public traded away certain parts of its natural rights that ordinary people were not exercising and in exchange got real benefits of more books being written`
\end{solution}





\savebox{\wkurl}{\descurl{http://digitalfree.info/w5}}
\chapter[Week 5. Digital lifestyle to support freedom. How can we contribute to the Free Digital Society?]{Digital lifestyle\\ to support freedom.\\ How can we contribute\\ to the Free Digital Society?}
\sethead{Week 5. Digital lifestyle to support freedom}
\section[Be careful which programs and services to use]{Be careful which programs\\ and services to use }
\begin{itemize}
 \item    Facebook is a monstrous surveillance engine.\index{Facebook}
\begin{itemize}
 \item        Facebook does surveillance on its useds and not useds as well. Call them "useds" rather than "users" because Facebook is using them, not vice versa.
 \item        Publishing somebody's photo in Facebook is treating him or her badly. Exception --- public events.
\end{itemize}
  \item   Distribute data in formats favorable to the free software.
\begin{itemize}
 \item        Ogg/Vorbis, Webm. NO MP*, NO Flash, NO quicktime and other non-free formats.\index{Data Formats!Ogg/Vorbis} \index{Data Formats!Webm} \index{Data Formats!Flash} \index{Data Formats!QuickTime}\index{Data Formats!MP*}
 \item        If your hardware records the data in non-free format convert files to the free formats before distributing.
 \item        Make sure the platform you use to distribute the files does not lead the users to use non-free software (e.g., YouTube has non-free JavaScript and Flash).\index{Youtube} \index{JavaScript}
\end{itemize}
  \item   Carefully select the programs and devices you use: avoid DRM, spyware, malware, proprietary jails.\index{Threats!DRM}
  \item   Don't use products that steal your freedom. E.g., Amazon "Swindle".\index{Swindle}
 \item    Mobile phones --- mobile computers with outstanding surveillance capabilities. Stalin's dream.
 \item    Protect your personal data --- it can be misused! (e.g., wi-fi access which requires your identity).
\end{itemize}

\section{Use free software yourself}
\begin{itemize}
 \item     Free GNU/Linux distributions:
\begin{itemize}
 \item         List of Free Linux Distributions: BLAG, gNewSense etc\index{GNU}\index{Software!GNU/Linux}
 \item         Well-known GNU/Linux distros continue to have non-free software!
\end{itemize}
 \item     Non-free parts in Linux: firmware. Binaries masqueraded as source-code.
 \item     Free GNU/Linux distributions contain only free software (hence, do not contain non-free firmware). As a result not all the hardware works.\index{Software!Firmware} \index{Software!Free}
 \item     Tolerate inconvenience if your aim is freedom. Otherwise you will lose your freedom.
 \item     Reject proprietary software. No matter how useful it is.\index{Software!Proprietary}
 \item     Prefer releasing no program to releasing proprietary program. Reason: releasing proprietary program harms society. If you do not release program at all you don't harm society at least.
 \item     Free Software Directory
\end{itemize}


\section{Education and Free Software}
\begin{itemize}
 \item    Four reasons to use free software in education:\index{Education}\index{School}\index{University}
\begin{enumerate}
 \item        \emph{Saving money.} Most superficial reason. Proprietary software companies started to give gratis or nearly gratis copies of their software to the educational institutions.
 \item        \emph{Educational system should not make society depending on the proprietary software.} Schools shall reject gratis copies of proprietary software for the same reasons they reject free samples of addictive drugs for the students.
 \item        \emph{Education of best programmers.} If software is proprietary student can't study its source code and education is not possible. Studying how to program is not studying how to write small programs. Students shall start from writing small parts in large programs. Free software gives this opportunity.
 \item        \emph{Ethical education. Education in citizenship.} Schools teach above all to the spirit of good will, habit to help the neighbor. Schools shall give examples of ethical behavior. Which is not possible with proprietary software.
\end{enumerate}
 \item    Students and teachers --- pressure your school to move to the free software:\index{Software!Free}
\begin{itemize}
 \item        Make the people aware that usage of non-free software is wrong.
 \item        If you are student and you have a class where proprietary software is used stand up in the very first lecture and ask to help you to find a way to make the work of this class with only free software. Explain that you are ready to work harder and understand possible inconveniences.
 \item        Your readiness to inconveniences shows that you are serious about freedom and influences other people around you.
\end{itemize}
\end{itemize}


\section{Make the Web Free}
\begin{itemize}
\item    The JavaScript Trap:
\begin{itemize}
 \item        Web servers load to your browser JavaScript programs. Typically they execute them silently --- without letting the users know.
 \item        Many of these programs are non-free
\end{itemize}
 \item    GNU LibreJS: Free JavaScript in your browser:\index{JavaScript}\index{Software!GNU LibreJS}
\addnom{JS}{JavaScript}
\begin{itemize}
 \item        Free add-on for Mozilla-based browsers (e.g., Firefox).
 \item        Checks all JavaScript code which is about to be executed by the browser. Blocks the code if it is non-trivial or non-free. Informs the user about blocked JavaScript in the page.
 \item        GNU LibreJS makes it easy to complain about non-free JavaScript to the web masters. It extracts web masters contact information from the page and shows a special "Complain" button to the user.
 \item        It's very important to complain about non-free JavaScript code to contribute to FSF cause.
\end{itemize}
 \item    The JavaScript Developers Task Force:
\begin{itemize}
 \item        To make sure that all the JavaScript executed in the browser is free is only a first step.
 \item        Need to be able to easily change the code and exchange it with others. Challenging task.
\end{itemize}
 \item    Switch to free browser: GNU IceCat:\index{Software!GNU IceCat}
\begin{itemize}
 \item        Mozilla-based.
 \item        Removed Firefox name and logo because they have imposed non-free restrictions.\index{Facebook}\index{Google}
 \item        Extensions list: modified in such a way that only free extensions are listed.
 \item        Does not send the addresses you've visited to google (privacy).
 \item        Tracking mechanisms are blocked (e.g., Facebook "Like" button, Google analytics etc).
 \item        LibreJS is active by default.\index{Software!GNU LibreJS}
\end{itemize}
\end{itemize}


\section{Act to support Freedom}
\begin{itemize}
 \item    Celebrate Freedom Fighters.
\begin{itemize}
 \item        Edward Snowden is a hero. Many people are trying to demonise him. It's important to celebrate him and the people like him.\index{Freedom Fighters}
\end{itemize}
 \item    Warn your friends about dangerous products.
\begin{itemize}
 \item        Example: Amazon Kindle. It makes people antisocial, divides them. Make sure your friends know this before they decide to use this product.\index{Swindle}
 \end{itemize}
 \item    Promote free software in your educational institution:
 \begin{itemize}
 \item        It's your responsibility to campaign for free software in your educational institution!\index{Education}
 \item        Setting up reverse engineering classes is very important. We need to operate the hardware with secret specifications without non-free software.
 \end{itemize}
 \item    Reject propaganda terms: \index{Threats!Propaganda}
 \begin{itemize}
 \item        E.g., "Piracy", "Theft". It is not right to call Sharing this way (see also lecture 2.4 "War on Sharing").\index{Piracy}\index{Sharing}
 \item        Good way to reject propaganda is to use jokes and humor. E.g., "Piracy? Attacking ships is very bad. Movie piracy? Pirates of the Caribbean is a good movie".
 \end{itemize}
 \item    Join Defective by Design movement:
 \begin{itemize}
 \item        More about Digital Restrictions Management (DRM) --- see in the lecture 2.4 "War on Sharing". \index{Threats!DRM}
 \item        This campaign shall be big. Don't expect that the battle will be won without you.
 \end{itemize}
 \item    Upgrade from Windows! It's not possible to live in freedom with Windows. Any version of Windows. Don't downgrade to Windows Vista at least.\index{Software!MS Windows}
 \item    Learn how to make easy changes: \index{Four Freedoms}\index{Easy Changes}
 \begin{itemize}
 \item        Anybody can exercise Freedoms 0 and 2 (see lecture 1.2) --- \emph{the freedom to run the program as you wish, for any purpose} and \emph{the freedom to redistribute copies so you can help your neighbor.}
 \item        Freedoms 1 and 3 (see lecture 1.2: \emph{the freedom to study how the program works, and change it so it does your computing as you wish} and \emph{the freedom to distribute copies of your modified versions to others}) require programming skills.\index{Four Freedoms}
 \item        It's useful to learn how to make easy changes in the software. You don't need to become software engineer in order to be able to do them. Analogy: it's useful to know how to make small maintenance of your car even if you are not a professional mechanic.
\end{itemize}
\end{itemize}

\section{Quiz 5}
\begin{question}[type=exam]
Dr. Stallman tells: "Linux in its usual version contains non-free software". What does he refer to? Select the best answer:
\begin{itemize}
\chk Non-free applications which can be found in the popular GNU/Linux distributions.
\chk Non-free firmware “blobs” which can be found in Linux kernel.
\chk Software included into most of the (non-free) GNU/Linux distributions which deals with proprietary data formats (e.g., Adobe Flash).
\end{itemize}
\end{question}
\begin{solution}
Non-free firmware “blobs” which can be found in Linux kernel.
\end{solution}


\begin{question}[type=exam]
Which of the following reasons was NOT stated among the reasons to use only free software in education?
\begin{itemize}
\chk Ethical education. Education in citizenship.
\chk Educational system should not make society depending on the proprietary software.
\chk Education system serves public interest therefore it shall not be favourable to any business.
\chk Education of best programmers.
\end{itemize}
\end{question}
\begin{solution}
Education system serves public interest therefore it shall not be favourable to any business.
\end{solution}


\begin{question}[type=exam]
Which features GNU LibreJS has? Select all answers that are correct:
\begin{itemize}
\chk It blocks JavaScript code if it is either non-trivial or non-free.
\chk It makes it easier to complain about non-free JavaScript to web masters.
\chk It collects information about most often blocked JavaScript non-free code and can send it to GNU LibreJS developes. The goal is to find most used non-free JavaScript code in the net.
\end{itemize}
\end{question}
\begin{solution}
\begin{itemize}
\item It blocks JavaScript code if it is either non-trivial or non-free.
\item It makes it easier to complain about non-free JavaScript to web masters.
\end{itemize}
\end{solution}

\begin{question}[type=exam]
Which features GNU IceCat has? Select all answers that are correct:
\begin{itemize}
\chk It blocks Google Analytics.
\chk It maintains list of free add-ons.
\chk It contains LibreJS extension to address the JavaScript problem known as "Javascript Trap".
\end{itemize}
\end{question}
\begin{solution}
\begin{itemize}
\item It blocks Google Analytics.
\item It maintains list of free add-ons.
\item It contains LibreJS extension to address the JavaScript problem known as "Javascript Trap".
\end{itemize}
\end{solution}

\begin{question}[type=exam]
You have video-recorded a public event. Your camera have saved video as MPEG4 file. What is the best strategy to distribute the video?
\begin{itemize}
\chk Publish the file I have got from the camera on any of the popular services for video hosting.
\chk Convert the file to Ogg Vorbis or webm format and find the service to publish it. This service shall deliver video to the users in the format favourable to the free software and should not lead its users to use non-free programs (e.g., non-free JavaScript).
\chk This video can not be distributed at all because it was recorded with non-free defected-by-design video camera.
\end{itemize}
\end{question}
\begin{solution}
Convert the file to Ogg Vorbis or webm format and find the service to publish it. This service shall deliver video to the users in the format favourable to the free software and should not lead its users to use non-free programs (e.g., non-free JavaScript).
\end{solution}


\begin{question}[type=exam]
Which of the following statements about free GNU/Linux distributions are WRONG?
\begin{itemize}
\chk They are well-known.
\chk They are more convenient than non-free GNU/Linux distributions.
\chk They do not contain non-free software, backdoors, DRM, malware and spyware inside.
\end{itemize}
\end{question}
\begin{solution}
\begin{itemize}
\item They are well-known.
\item They are more convenient than non-free GNU/Linux distributions.
\end{itemize}
\end{solution}


\begin{question}[type=exam]
Why is it important to celebrate freedom fighters? Select the best answer:
\begin{itemize}
\chk We need to show to the Big Brother that we are careful about freedom and respect those who are fighting for it.
\chk There people and powerful organizations which are trying to demonise freedom fighters.That's why it's especially important to state that they are heroes who are fighting for our freedom.
\chk We want them to realize that they have done really great things for us. It's our moral obligation to say "thank you!" to our heroes.
\end{itemize}
\end{question}
\begin{solution}
There people and powerful organizations which are trying to demonise freedom fighters.That's why it's especially important to state that they are heroes who are fighting for our freedom.
\end{solution}


\begin{question}[type=exam]
How free software helps to teach programming? Select all the answers that are correct:
\begin{itemize}
\chk It makes it possible to study source code.
\chk It makes it possible to write small changes in the big programs.
\chk By studying free software source code the student can know what is bad coding style and how he or she shall not write programs.
\end{itemize}
\end{question}
\begin{solution}
\begin{itemize}
\item It makes it possible to study source code.
\item It makes it possible to write small changes in the big programs.
\item By studying free software source code the student can know what is bad coding style and how he or she shall not write programs.
\end{itemize}
\end{solution}


\begin{question}[type=exam]
What is "Defective By Design" about? Select the best answer:
\begin{itemize}
\chk Dr. Stallman names Windows operating system in such a way
\chk Name for all non-free software used by FSF. This software is defective by design because it is designed to exercise unjust power over the users.
\chk Campaign exposing DRM-encumbered devices and media for what they really are. Supporters are working together to eliminate DRM as a threat to innovation in media, the privacy of readers, and freedom for computer users.
\end{itemize}
\end{question}
\begin{solution}
Campaign exposing DRM-encumbered devices and media for what they really are. Supporters are working together to eliminate DRM as a threat to innovation in media, the privacy of readers, and freedom for computer users.
\end{solution}



\begin{question}[type=exam]
What is proprietary jail? Select the best answer:
\begin{itemize}
\chk US prison operated by private company. 
\chk Proprietary systems that are jails --- they do not allow the user to freely install applications. These systems are platforms for censorship imposed by the company that owns the system. 
\chk FreeBSD jail mechanism (implementation of operating system-level virtualization that allows administrators to partition a system into several independent mini-systems called jails) used in some of Android and Apple devices. 
\end{itemize}
\end{question}
\begin{solution}
Proprietary systems that are jails --- they do not allow the user to freely install applications. These systems are platforms for censorship imposed by the company that owns the system.
\end{solution}





\appendix

\renewcommand{\sethead}[1]{\markboth{\MakeUppercase{#1}}{}\renewcommand{\mymark}{\MakeUppercase{\appendixname}~A}}

\addcontentsline{toc}{part}{Appendices}
\savebox{\wkurl}{}
\chapter{Answers to quizzes}
\sethead{Answers to quizzes}

\section{Quiz 1}
\printsolutions[chapter={1}]

\section{Quiz 2}
\printsolutions[chapter={2}]

\section{Quiz 3}
\printsolutions[chapter={3}]

\section{Quiz 4}
\printsolutions[chapter={4}]

\section{Quiz 5}
\printsolutions[chapter={5}]

\chapter{List of Abbreviations}
\sethead{List of Abbreviations}
\cleardoublepage

%% We don't need \chapter to be issued by \printaddnom. Redefined addnom environment
\makeatletter
\def\thenomenclature{%
  \nompreamble
  \list{}{%
    \labelwidth\nom@tempdim
    \leftmargin\labelwidth
    \advance\leftmargin\labelsep
    \itemsep\nomitemsep
    \let\makelabel\nomlabel}}
\makeatother

\printnomenclature[1.2cm]

\chapter{Index}
\sethead{Index}
\cleardoublepage

%% Don't want \chapter or \section to be generated by \printindex
\renewcommand{\section}[1]{}
\renewcommand{\chapter}[1]{}
\renewcommand{\cleardoublepage}{}
\renewcommand{\indexname}{\vspace*{-1cm}}
\printindex

\end{document}
