\section{Quiz 1}
\begin{question}[type=exam]
What is the most important question to judge the program?
\begin{itemize}
\chk How much does it cost to me (what is its total cost of ownership)?
\chk What does it do for my freedom?
\chk How buggy is it?
\end{itemize}
\end{question}
\begin{solution}
What does it do for my freedom?
\end{solution}

\begin{question}[type=exam]
If the program is not free we (FSF) call it ...
\begin{itemize}
\chk  Shareware.
\chk Proprietary.
\chk SaaSS.
\end{itemize}
\end{question}
\begin{solution}
Proprietary.
\end{solution}

\begin{question}[type=exam]
What is NOT one of the four freedoms?
\begin{itemize}
\chk Run the program as you wish.
\chk Freedom to redistribute the program copies.
\chk Freedom to describe flaws in the program.
\end{itemize}
\end{question}
\begin{solution}
Freedom to describe flaws in the program.
\end{solution}

\begin{question}[type=exam]
What is the meaning of the acronym GNU?
\begin{itemize}
\chk It it not an acronym. It's another name for the wildebeest.
\chk GNU's Not Unix.
\chk Good News Unlimited.
\end{itemize}
\end{question}
\begin{solution}
GNU's Not Unix.
\end{solution}

\begin{question}[type=exam]
Why Free Software Foundation (FSF) promotes the name "GNU/Linux" contrary to just "Linux"?
\begin{itemize}
\chk The system some call "Linux" is the GNU system, combined with the kernel Linux.
\chk FSF wants GNU to be associated with Linux in order to get more power in protecting Software Freedoms.
\chk The question is wrong. FSF does not promote this name.
\end{itemize}
\end{question}
\begin{solution}
The system some call "Linux" is the GNU system, combined with the kernel Linux.
\end{solution}

\begin{question}[type=exam]
Are all the GNU/Linux software distributions free?
\begin{itemize}
\chk Yes. They are all free as in freedom. But some of them can be on sale because freedom is not about price at all.
\chk No. There is only one free GNU/Linux software distribution in the world and it is distributed by FSF. All other GNU/Linux software distros contain proprietary components and can not be defined as free.
\chk No. There are few free GNU/Linux software distributions. FSF promotes them.
\end{itemize}
\end{question}
\begin{solution}
No. There are few free GNU/Linux software distributions. FSF promotes them.
\end{solution}


\begin{question}[type=exam]
You wrote a program and in the process implemented an idea. After publishing your application you found that this idea had been patented previously (before you started to even think about it). What are the dangers?
\begin{itemize}
\chk No danger because you have invented this idea independently from the patent owner.
\chk You can be sued by the patent owner. Your users can be sued also.
\chk You have committed a criminal offense described in the US penal code. There is a risk of being arrested by US police when you cross the USA border.
\end{itemize}
\end{question}
\begin{solution}
You can be sued by the patent owner. Your users can be sued also.
\end{solution}

\begin{question}[type=exam]
What a supporter of free software is expected to think about an idea to use Skype as a collaboration tool and to put the Skype user id in his or her email signature?
\begin{itemize}
\chk Sounds cool. I want to communicate with the people via Skype, a lot of my friends use it. I want them to know my Skype credentials as they will be able to use Skype to reach me.
\chk This idea is very bad. Skype is proprietary software and using it is a threat for the freedom. By putting my Skype username in the signature I will do even worse --- I will promote the usage of the non-free software among the people I cooperate with.
\chk This idea sounds bad. Skype is owned by Microsoft and this corporation should be punished for their unethical behaviour and creation of the non-free software.
\end{itemize}
\end{question}
\begin{solution}
This idea is very bad. Skype is proprietary software and using it is a threat for the freedom. By putting my Skype username in the signature I will do even worse --- I will promote the usage of the non-free software among the people I cooperate with.
\end{solution}

\begin{question}[type=exam]
What are the malicious features of Adobe Flash player?
\begin{itemize}
\chk Digital Restrictions Management (DRM) and user surveillance feature where one site was able to write data into the Flash Player and another site could interrogate the Flash Player thus enabling sites to cross-identify a user.
\chk Digital Restrictions Management (DRM) and adware.
\chk Adobe Flash Player constantly reports the user's IP address to the Adobe headquarters.
\addnom{IP}{Internet Protocol}
\end{itemize}
\end{question}
\begin{solution}
Digital Restrictions Management (DRM) and user surveillance feature where one site was able to write data into the Flash Player and another site could interrogate the Flash Player thus enabling sites to cross-identify a user.
\end{solution}

\begin{question}[type=exam]
Imagine that in order to promote the progress of symphonic music the European governments of XVIII century established a patent system on music themes. How would it influence the work of composers?
\begin{itemize}
\chk Composers would become richer and would be able to be more creative because of the need to invent the new themes in the music all the time.
\chk No influence at all.
\chk It would be harder for composers to create rich symphonies because every new piece of art reuses the ideas developed by previous artists in one way or the other
\end{itemize}
\end{question}
\begin{solution}
It would be harder for composers to create rich symphonies because every new piece of art reuses the ideas developed by previous artists in one way or the other
\end{solution}


