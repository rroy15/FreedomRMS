
\newcommand{\myhref}[2]{#2}
\definecolor{light-gray}{gray}{0.97}

\pgfdeclarelayer{background}
\pgfsetlayers{background,main}
\begin{center}
\rotatebox{90}{\resizebox{!}{.7\textheight}{\begin{tikzpicture}[mindmap,
every node/.style={concept,execute at begin node=\hskip0pt},
root concept/.append style={
concept color=black, fill=white, line width=1ex, text=black,minimum size=3cm,text width=3cm,font=\large\scshape},
text=black,
concept color=light-gray, grow cyclic,
level 1/.append style={level distance=3.7cm,sibling angle=90},
level 2/.append style={level distance=3cm,sibling angle=45},
level 3/.append style={level distance=2.3cm,sibling angle=45},
remark/.style={
rectangle,
sloped,
minimum size=4mm,
very thin,
color=white,
text=black,
font=\itshape
}]
\node[root concept] (root) {Human rights in computing}
  child [concept] {node {\myhref{https://www.gnu.org/philosophy/free-software-intro.html}{Free software movement}}
  	child [concept] {node {Freedom for people using software}}
	child [concept] {node (fsf) {\myhref{https://groups.google.com/forum/\#!msg/net.unix-wizards/8twfRPM79u0/1xlglzrWrU0J}{Start: September, 1983}}}
  }
 child [concept] {node (gnulnx) {\myhref{https://www.gnu.org/gnu/linux-and-gnu.html}{GNU/Linux}}
  	child [concept] {node {\myhref{http://en.wikipedia.org/wiki/Unix}{UNIX}-like system}}
	child [concept] {node {\myhref{http://www.gnu.org/}{GNU's Not Unix}}}
	child [concept] {node {1992: First free OS}
		child [concept] {node {Almost all OS is ready}}
		child [concept] {node {\myhref{http://web.archive.org/web/20110721105526/http://www.kernel.org/pub/linux/kernel/Historic/old-versions/RELNOTES-0.12}{Linux kernel became free}}}
	}
	child [concept] {node {Nonfree sw on top of GNU/Linux limits your freedom}
		child [concept] {node {Most \myhref{http://www.gnu.org/distros/}{distros} contain nonfree sw}}
	}
  }
  child [concept] { node (fourfr) {Four Freedoms}
	child [concept] { node {3: Distribute modified versions}}
	child [concept] { node {2: Redistribution}
       		child [concept] {node {Basic moral rule - help your friends}}
        }
	child [concept] { node {1: Study and change the sources}}
        child [concept] { node {0: Run the program as you wish}}
  }
  child [concept] { node {How to judge the program?}
	child [concept] {node (freesw) {\myhref{https://www.gnu.org/philosophy/free-sw.html}{Free software}}
		child [concept] {node {User controls the program}}
	}
        child [concept] { node {What it does for my freedom?}
		child [concept] { node {\myhref{https://www.gnu.org/philosophy/selling.html}{"Free" is not about price at all}}}
	}
	child [concept] {node {Nonfree software}
		child [concept] {node {Program controls the user}}
		child [concept, sibling angle=45] {node {Examples: Gratis but nonfree}
			child [concept, sibling angle=45] {node {Adobe Flash Player}
				child [concept] {node {DRM: restricts the user}}
				child [concept] {node {Surveillance: Super-Cookies}}
				child [concept] {node {Malicious}}
			}
			child [concept, sibling angle=45] {node {\myhref{http://stallman.org/skype.html}{Skype}}
				child [concept] {node {Promotes nonfree software}}
			}
			child [concept, sibling angle=45] {node {\myhref{http://www.gnu.org/philosophy/proprietary-surveillance.html}{Angry Birds}}}
		}
	}
  };
 \begin{pgfonlayer}{background}
  	\draw [concept connection] (fourfr) edge node[remark,above=3.5pt] {Gives} node [remark,below=3.5pt] {Four Freedoms} (freesw);
	\draw [concept connection] (fsf) edge node[remark,above=3.5pt] {Goal: develop} node [remark,below=3.5pt] {entirely free OS} (gnulnx);
  \end{pgfonlayer}
\end{tikzpicture}}}
\end{center}

