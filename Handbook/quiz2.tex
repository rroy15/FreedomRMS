\section{Quiz 2}

\begin{question}[type=exam]
Is it a good or bad thing to participate in the digital society? Select one answer.
\begin{itemize}
\chk Yes, sure. New digital technologies improve people's living conditions dramatically and we should do our best to let everybody to benefit from the new technologies.
\chk No, not at all. New digital technologies make the people less smart, they restrict abilities of the human brain to think. We need to avoid them and focus on developing our brains instead.
\chk It depends on how digital society is set up and how does it respect people's freedom.
\end{itemize}
\end{question}
\begin{solution}
It depends on how digital society is set up and how does it respect people's freedom.
\end{solution}

\begin{question}[type=exam]
Which of the following is NOT an example of surveillance threat to the people's freedom?
\begin{itemize}
\chk Systems to record the user's location using the data from GSM or CDMA base stations where user's phone was registered.
\chk Portable GPS tracker gadget which tracks your geo-location and stores it in its memory. GPS track can be downloaded from the gadget through USB or RS-232 interfaces.
\chk Companies monitoring the behavior of their users via web sites (e.g., cookies) for advertising purposes.
\end{itemize}
\end{question}
\begin{solution}
Portable GPS tracker gadget which tracks your geo-location and stores it in its memory. GPS track can be downloaded from the gadget through USB or RS-232 interfaces.
\end{solution}

\begin{question}[type=exam]
Several examples of censorship in different countries were presented in the video. Which country was NOT mentioned in this context?
\begin{itemize}
\chk Turkey
\chk Finland
\chk Saudi Arabia
\end{itemize}
\end{question}
\begin{solution}
Saudi Arabia
\end{solution}

\begin{question}[type=exam]
What are "digital handcuffs"?
\begin{itemize}
\chk Modern handcuffs for the prisoners with escape records. They have surveillance functionality implemented using CDMA technology.
\chk The situation when software users are restricted by the secret data format which prevents data interchange with some of the other programs.
\chk The situation when the user is forced to use the proprietary software because there is no free software with the similar functionality available.
\end{itemize}
\end{question}
\begin{solution}
The situation when software users are restricted by the secret data format which prevents data interchange with some of the other programs.
\end{solution}

\begin{question}[type=exam]
"Software as a Service Substitute" means losing the control on your computing as it is done on somebody's else server. What is the difference between Wikipedia and Speech Recognition services from end-user point of view?
\begin{itemize}
\chk No difference. Both process user's data on their side and both should be avoided because the users is loosing the control on his/her computing.
\chk By contributing to Wikipedia the user helps it to do its computing and this is expected by the user. By using Speech Recognition service user's own computing is done by the server.
\chk Wikipedia is a service to share the data with other users in a free way. Speech Recognition service is proprietary solution therefore it is bad idea to use it.
\end{itemize}
\end{question}
\begin{solution}
By contributing to Wikipedia the user helps it to do its computing and this is expected by the user. By using Speech Recognition service user's own computing is done by the server.
\end{solution}

\begin{question}[type=exam]
What is mr. Stallman's opinion on the piracy? Select the best answer.
\begin{itemize}
\chk Attacking ships is bad. But it has nothing in common with sharing which is good. It's morally bad to try to associate something which is good (sharing) with something which is bad (attacking ships).
\chk Piracy is a good thing and we should all be pirates. It is in the nature of the human being to share things.
\chk Piracy is sometimes good and sometimes bad. It depends on the nature of the works which are shared (e.g., sharing of the fiction books can be sometimes considered bad while sharing of the engineering books is always good).
\end{itemize}
\end{question}
\begin{solution}
Attacking ships is bad. But it has nothing in common with sharing which is good. It's morally bad to try to associate something which is good (sharing) with something which is bad (attacking ships).
\end{solution}

\begin{question}[type=exam]
What Precarity threat is about?
\begin{itemize}
\chk Everybody needs to be educated about the dangers of the digital society but in reality only the minor percentage of the population has this knowledge.
\chk Any activity in the Internet requires cooperation from a lot of stakeholders --- ISP, hosting providers etc. Governments can use these stakeholders to disconnect the service (e.g., web site) they don't like from the Internet by putting pressure on those stakeholders. It is possible because a lot of service providers have a freedom to break the contract with the user in any time.
\chk Children are becoming the victims of the violent content accessible in the Internet. As a result, school shooting occur. New policy towards the way children can use Internet should be developed.
\end{itemize}
\end{question}
\begin{solution}
Any activity in the Internet requires cooperation from a lot of stakeholders --- ISP, hosting providers etc. Governments can use these stakeholders to disconnect the service (e.g., web site) they don't like from the Internet by putting pressure on those stakeholders. It is possible because a lot of service providers have a freedom to break the contract with the user in any time.
\end{solution}

\begin{question}[type=exam]
What is "universal backdoor"? Select one answer.
\begin{itemize}
\chk The interface through which somebody else than the user can install any software changes in the user's system without user's permission.
\chk Method of bypassing normal authentication, securing unauthorized remote access to a computer, obtaining access to plaintext, and so on, while attempting to remain undetected.
\chk The name of the method to unencrypt DRM-protected movies in some of DVD players.
\end{itemize}
\end{question}
\begin{solution}
The interface through which somebody else than the user can install any software changes in the user's system without user's permission.
\end{solution}

\begin{question}[type=exam]
Which parts of "dirty tricks" campaign against WikiLeaks was discussed in the lectures?
\begin{itemize}
\chk CIA used sexual entrapment against Julian Assange.
\chk US government tried to convince the companies which provided services to WikiLeaks (e.g., eCommerce services) that it's in their best interests to stop serve WikiLeaks.
\chk Governments run defamation campaign against WikiLeaks trying to portrait it as untrustworthy source of information.
\end{itemize}
\end{question}
\begin{solution}
US government tried to convince the companies which provided services to WikiLeaks (e.g., eCommerce services) that it's in their best interests to stop serve WikiLeaks.
\end{solution}

\begin{question}[type=exam]
Why SaaSS is an equivalent of proprietary malware? Select one answer.
\begin{itemize}
\chk If you use SaaSS your computing is done by the server. Server owner can misuse your data.
\chk SaaSS has the same effect as running non free programs with universal backdoor (because server owner can change server software in any time and it changes the way user's computing is done).
\chk SaaSS is proprietary solution. Proprietary solution is always malware.
\end{itemize}
\end{question}
\begin{solution}
SaaSS has the same effect as running non free programs with universal backdoor (because server owner can change server software in any time and it changes the way user's computing is done).
\end{solution}


